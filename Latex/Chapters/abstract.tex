\begin{abstract}
    The growing awareness among people, including governments and individuals, of the risks posed by the pollution of gasoline and diesel vehicle emissions has contributed significantly to the shift toward clean energy for both public and private transport. Efficient and accessible transportation is a critical challenge that affects urban planners and corporate decision makers. Significant investments in technology and infrastructure are necessary to ensure that sufficient charging stations are available in strategic locations. This is crucial to ensure a successful transition to sustainable electric transport.
    
    This thesis explores the use of the multi-objective evolutionary algorithm to optimize the placement and configuration of electric vehicle charging stations (EVCS). The goal is to find a solution that maximizes the geographic area covered by EV charging stations, minimizes the cost of setting up the infrastructure, and maximizes the power level of the stations to improve charging efficiency and reduce waiting times. By applying MOOP, such as the NSGA-II algorithm, we demonstrate the potential of evolutionary algorithms to address the complexities of EV charging station layout in an urban environment. The results show that NSGA-II provides an efficient and versatile solution to the challenges of the electric vehicle charging station infrastructure.
    
\end{abstract}