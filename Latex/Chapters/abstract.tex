\begin{abstract}
The increasing awareness of the environmental impact of gasoline and diesel vehicle emissions has contributed to the global shift toward cleaner energy in both public and private transportation. Ensuring efficient and accessible transport remains a critical challenge for urban planners and decision-makers. To support this transition, significant advancements in technology and infrastructure are required—particularly in developing an effective network of electric vehicle charging stations (EVCS) at strategically chosen locations.

This thesis explores the use of the multi objective evolutionary algorithm to optimize the placement and configuration of electric vehicle charging stations (EVCS). The goal is to find a solution that maximizes the geographic area covered by EV charging stations, minimizes the cost of setting up the infrastructure, and maximizes the power level of the stations to improve charging efficiency and reduce waiting times. By applying a multi-objective optimization process (MOOP) using a multi-objective evolutionary algorithm (MOEA), such as the NSGA-II algorithm, we demonstrate the potential of evolutionary algorithms to address the complexities of EV charging station layout in an urban environment. The results show that NSGA-II provides an efficient and versatile solution to the challenges of the electric vehicle charging station infrastructure.

The NSGA-II algorithm found optimized EVCS layouts that balance five key objectives: (1) maximize coverage, (2) maximize charger speed to reduce waiting time, (3) minimize the number of stations, (4) minimize the number of chargers, and (5) minimize average distance, all aimed at minimizing total infrastructure cost and improving efficiency. The optimized solution reduced the network cost from \$5,867,800 (about 5.87 million) to \$3,750,000, achieving a 36\% decrease. This improvement was achieved by reducing the number of EVCS from 19 to 7 and the total number of chargers from 69 to 18.


However, because this study used a multi-objective optimization approach, the algorithm balanced several competing goals rather than focusing on a single objective. As a result, coverage per station decreased from 11.89 to 3.33 on average. This reduction reflects trade-offs made to prioritize other objectives, especially minimizing the average distance between electric vehicles (EVs) and their nearest charging station. In this respect, the solution was effective, as the average user travel distance improved slightly, decreasing from 17.39 km to 17.37 km, which enhances accessibility and reduces user inconvenience.

Overall, the NSGA-II approach demonstrated the ability to generate a variety of balanced solutions, supporting the development of cost-effective, scalable, and accessible electric vehicle charging infrastructure.
\end{abstract}