
The development of electric vehicle charging infrastructure has become a critical issue due to the rapid adoption of electric vehicles (EVs) and the growing demand for sustainable transportation. Solving this problem requires addressing multiple conflicting objectives such as coverage, charging speed, number of stations, and accessibility. As a result, researchers have increasingly turned to Multi-Objective Optimization Problems (MOOP) and Evolutionary Algorithms (EAs), particularly Genetic Algorithms (GAs) and the Non-dominated Sorting Genetic Algorithm II (NSGA-II), to identify optimal solutions that strike a balance among these objectives.

\subsection{Multi-objective Approaches in EVCS Planning}

The EVCS deployment problem is naturally suited for multi-objective modeling due to the need to balance trade-offs such as cost, coverage, and user convenience.\cite{A multi-objective optimization model for electric vehicle charging station location planning} proposed a multi-objective optimization model that minimizes installation costs and user travel distance while maximizing station utilization. Their study highlighted the importance of Pareto-based solution sets in evaluating trade-offs and supporting planning decisions. Similarly,\cite{A genetic algorithm-based optimization for the location of electric vehicle charging stations} applied a genetic algorithm to optimize charging station locations by minimizing operational costs and maximizing service coverage, further demonstrating the effectiveness of evolutionary approaches in capturing the complexities of EVCS planning.

\subsection{Genetic Algorithms and NSGA-II in EV Infrastructure Planning}

Genetic Algorithms have been widely adopted in the field due to their robust global search capabilities. \citet{A genetic algorithm-based optimization for the location of electric vehicle charging stations} developed a GA-based model that optimized EVCS placement in urban environments by considering cost and traffic congestion, revealing the GA’s suitability for handling real-world constraints and objective interactions.

NSGA-II has also gained attention for its ability to generate diverse Pareto-optimal solutions in multi-objective scenarios. \citet{A multi-objective optimization model for electric vehicle charging station location planning} utilized NSGA-II to optimize EVCS locations based on three objectives: minimizing user travel time, minimizing power grid losses, and maximizing user satisfaction. Their findings showed that NSGA-II is effective in producing non-dominated solution sets, allowing planners to select trade-offs based on practical requirements.

\subsection{Key Optimization Objectives in EVCS Research}

Typical optimization goals in EVCS planning include maximizing geographic coverage, minimizing user inconvenience (e.g., travel distance or wait time), reducing infrastructure and operational costs, and improving energy efficiency. For instance, the GA-based model in \cite{A genetic algorithm-based optimization for the location of electric vehicle charging stations} focused on minimizing costs while expanding access across high-demand areas. Meanwhile, \cite{A multi-objective optimization model for electric vehicle charging station location planning} integrated objectives that directly relate to user experience and grid impact, reflecting a holistic approach to infrastructure deployment.

However, objectives such charger speed, number of stations, and number of chargers are less frequently addressed in combination. Existing studies tend to optimize a subset of these objectives, which may limit their applicability in scenarios that demand a more comprehensive planning approach.

\subsection{Comparative Insights and Research Gaps}

Although the reviewed studies provide valuable insights, several research gaps remain. Neither \cite{A multi-objective optimization model for electric vehicle charging station location planning} nor \cite{A genetic algorithm-based optimization for the location of electric vehicle charging stations} consider all five objectives investigated in this research: maximizing coverage, maximizing charger speed (or minimizing wait time), minimizing number of stations, minimizing number of chargers, and minimizing average distance. Charger speed, in particular, is often overlooked as a decision variable, even though it directly impacts user satisfaction and throughput.

Additionally, while \citet{A multi-objective optimization model for electric vehicle charging station location planning} successfully applied NSGA-II to a multi-objective framework, its application on large-scale or real-world datasets incorporating urban mobility patterns and service equity remains underexplored. Furthermore, neither study explicitly addresses access disparities across different population groups, a growing concern in EV infrastructure planning.

\subsection{Summary}

Both studies reviewed demonstrate the value of using evolutionary algorithms—specifically Genetic Algorithms and NSGA-II—for optimizing EV charging infrastructure. They effectively manage complex trade-offs among objectives like cost, distance, and coverage. 

However, comprehensive frameworks that simultaneously address broader objectives such as charger speed, infrastructure minimization, and average user distance are limited. This study addresses these gaps by employing NSGA-II within a five-objective optimization framework to improve the strategic planning of EVCS deployment.


