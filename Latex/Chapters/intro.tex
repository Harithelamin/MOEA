The rapid growth in electric vehicle (EV) adoption offers significant benefits, including reduced air pollution and decreased reliance on traditional fuels like diesel and gasoline. However, this transition also presents challenges for urban transportation systems, particularly the need to develop reliable and accessible charging infrastructure.

Selecting locations for electric vehicle charging stations is a complex issue that requires balancing several conflicting objectives. With the increasing adoption of electric vehicles globally, creating a reliable, efficient, and accessible infrastructure has become critical. A key challenge in this process is identifying the optimal locations for charging stations to meet the needs of electric vehicle users while minimizing associated costs and maximizing coverage to ensure easy access.

Minimizing overall infrastructure costs is an important goal when selecting the location of charging stations. This includes both direct costs, such as installation and maintenance of the stations, as well as the costs associated with users accessing those stations. Reducing these costs is a critical aspect of urban planning, making it essential to strategically distribute charging stations to keep expenses low while still meeting demand. Efficient distribution plays a key role in balancing affordability with user accessibility.

In addition to cost considerations, minimizing the travel distance for electric vehicle (EV) users is another crucial factor in determining the location of charging stations. The availability of charging stations directly influences the feasibility of using EVs. Stations should be strategically located, both in densely populated urban areas and in rural regions where access to charging infrastructure may be limited. Reducing the distance drivers must travel to find a charging point is essential, as it encourages EV adoption. On the other hand, long wait times at charging stations or difficulty finding a nearby station can deter drivers from using EVs altogether.

Furthermore, speed of cahrgers should be considered to minimize charging time. Optimizing the charging power at each station enhances the overall efficiency of the network, leading to faster charging times and a better user experience. This factor is crucial in ensuring that the charging infrastructure can meet the growing demand for electric vehicles while minimizing user wait times.

In contrast, traditional approaches to solving the EVCS location problem typically rely on optimization techniques such as mathematical programming. These methods often focus on single-objective optimization, prioritizing one aspect of the problem, such as minimizing cost. While effective in certain scenarios, they often fail when addressing trade-offs between multiple objectives, such as cost, coverage, and load balancing. Additionally, traditional optimization techniques can be computationally expensive and time-consuming, especially when dealing with large urban areas and numerous potential charging station locations.

Multi-objective optimization techniques, such as multi-objective evolutionary algorithms (MOEAs), have emerged as a promising alternative to address the complexity of the EVCS problem. These algorithms are designed to explore a range of solutions that represent different trade-offs between competing objectives. By incorporating multiple objectives into the optimization process, MOEAs provide decision makers with a broader set of possible solutions, enabling them to select the solution that best balances the various costs and benefits of charging station placement.

These methods are particularly valuable when solving real-world problems that require the simultaneous optimization of multiple, often conflicting, objectives. In conclusion, locating electric vehicle charging stations is a complex task that involves addressing several objectives, such as maximizing coverage and optimizing charger power to reduce waiting times. Traditional methods may fall short when managing these competing objectives, making multi-objective evolutionary algorithms an attractive solution for optimizing EV charging infrastructure placement.

\section{Motivation}

The transition to electric vehicles (EVs) is a crucial step toward sustainable transportation, helping to reduce air pollution, greenhouse gas emissions, and reliance on fossil fuels such as gasoline and diesel. As EV adoption accelerates globally, the need for a reliable, accessible, and efficient charging infrastructure becomes increasingly important. 

Designing the optimal locations and configurations for electric vehicle charging stations (EVCS) is a complex problem involving multiple, often conflicting objectives. Key goals include maximizing geographic coverage to ensure accessibility for users, enhancing charger speeds to reduce waiting and charging times, minimizing infrastructure costs by limiting the number of stations and chargers, and reducing the average travel distance for EV users to access charging points. Balancing these objectives is essential to foster widespread EV adoption and create a user friendly charging network.


Traditional optimization methods are often inadequate for solving such complex, multidimensional problems. This is where multi-objective evolutionary algorithms (MOEAs), such as NSGA-II, become valuable. These algorithms are well-suited to handle the complexities of EVCS infrastructure optimization, as they can search for solutions that simultaneously satisfy multiple objectives.

The motivation behind this thesis arises from the need to explore advanced optimization techniques like MOEAs to address the challenges faced by urban planners and decision-makers in the development of electric vehicle charging infrastructure. As the number of electric vehicles grows, the demand for a robust charging network increases. The placement and configuration of charging stations must ensure adequate coverage across urban areas while minimizing infrastructure costs.

Furthermore, this research aims to demonstrate how evolutionary algorithms, particularly NSGA-II, can provide efficient and scalable solutions to these challenges, contributing to the creation of a sustainable and efficient electric vehicle charging infrastructure.

\section{Problem Definition}

The rapid trend of governments pushing for electric vehicles has created an urgent need for a strong and widespread electric vehicle charging infrastructure. Ensuring that electric vehicle charging stations are optimally distributed across geographies, with high capacity and cost-effectiveness to meet the needs of electric vehicle users, is a complex challenge. The problem is multifaceted, involving several objectives, such as increasing coverage area, minimizing costs, and reducing waiting time. This research focuses on leveraging multi-objective evolutionary algorithms to address these objectives and find optimal solutions for the deployment of electric vehicle charging infrastructure. The problem can be formalized as follows:

\begin{itemize}
    \item \textbf{Maximize Charger Speed:}The efficiency of charging stations plays a key role in determining the speed at which electric vehicles can be recharged. In general, stations with higher power output can reduce charging durations, potentially leading to shorter waiting times for users. These improvements may enhance the overall user experience and contribute to a more efficient operation of the charging network. As the adoption of electric vehicles increases, the availability of high-speed charging infrastructure becomes an important factor in meeting user needs.
    
    \item \textbf{Maximize Coverage:} Expanding the geographic coverage of electric vehicle charging stations is essential for improving accessibility for users. A well-distributed network can allow electric vehicle owners to find and reach charging points more easily, potentially reducing the need for long detours or extended travel times. Improved accessibility could enhance user convenience and may play a significant role in promoting the adoption of electric vehicles. As more areas are equipped with reliable charging infrastructure, range anxiety could be alleviated, making the ownership of electric vehicles more practical and supporting the transition to a more sustainable transportation system.


    
    \item \textbf{Maximize Charger Speed:} The efficiency of charging stations plays a key role in determining the speed at which electric vehicles can be recharged. In general, stations with higher power output can reduce charging durations, potentially leading to shorter waiting times for users. These improvements may enhance the overall user experience and contribute to a more efficient operation of the charging network. As the adoption of electric vehicles increases, the availability of high-speed charging infrastructure becomes an important factor in meeting user needs.
    
    \item \textbf{Minimize the Number of Stations:} Reducing the number of charging stations, without compromising network coverage, is essential for lowering costs and enhancing operational efficiency. A careful balance between station location and network coverage ensures that the service area is adequately covered while minimizing infrastructure costs.
    
    \item \textbf{Minimize the Number of Chargers:} In addition to reducing the number of charging stations, it is important to minimize the number of individual chargers at each station. This approach helps lower initial investment and operating costs, while still maintaining service levels and user convenience.    

\end{itemize}

However, the economic feasibility of building or expanding electric vehicle charging infrastructure is a significant consideration for decision-makers. Achieving a balance between wide geographic coverage and cost-effectiveness is essential to enhancing system performance. This challenge becomes even more critical when coupled with the need to increase charger speeds, which can reduce charging and waiting times and improve overall efficiency.

While deploying a large number of charging stations across broad areas can enhance accessibility, it demands substantial financial investment. This underscores the need for strategic planning that ensures both affordability and adequate service coverage for electric vehicle users.

\section{Objective of the Thesis}
This thesis aims to:
\begin{itemize}
    \item Investigate multi-objective evolutionary algorithms (MOEAs) to solve the EVCS infrastructure problem.
    \item Optimize multiple objectives simultaneously, such as coverage, number of EVCS, number of chargers, charger speed (with the goal of upgrading all chargers to Level 3 fast chargers), and the average distance between EVs and EVCS.
    \item Present and analyze the results from the research experiments conducted in this thesis, and compare them with available results from previous research.
\end{itemize}