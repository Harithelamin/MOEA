Rapid growth in the adoption of electric vehicles (EVs) offers significant benefits, including reduced air pollution and decreased reliance on traditional fuels such as diesel and gasoline. However, this transition also presents challenges for urban transportation systems, particularly the need to develop reliable and accessible charging infrastructure.

The selection of locations for electric vehicle charging stations is a complex issue that requires balancing several conflicting objectives. With the increasing adoption of electric vehicles globally, creating reliable, efficient, and accessible infrastructure has become critical. A key challenge in this process is identifying the optimal locations for charging stations to meet the needs of electric vehicle users while minimizing associated costs and maximizing coverage to ensure easy access.



Minimizing overall infrastructure costs is a primary goal when selecting charging station locations. This includes direct costs, such as station installation and maintenance, as well as costs associated with user access. Reducing these costs is a critical aspect of urban planning, making it essential to strategically distribute charging stations to keep expenses low while still meeting demand. Effective distribution plays a key role in balancing costs and user accessibility.

In addition to cost considerations, minimizing travel distance for EV users is a critical factor in determining charging station locations. The availability of charging stations directly impacts the feasibility of EV use. Stations should be strategically located, whether in densely populated urban areas or in rural regions where access to charging infrastructure may be limited. Reducing the distance drivers must travel to find a charging station is essential, as it encourages the adoption of EVs.

Furthermore, the speed of chargers should be considered to minimize charging time. Optimizing the charging power at each station enhances the overall efficiency of the network, leading to faster charging and a better user experience. This factor is crucial to ensuring that the charging infrastructure can meet the growing demand for electric vehicles while minimizing user wait times.

In contrast, traditional approaches to solving the EVCS location problem typically rely on optimization techniques such as mathematical programming. These methods often focus on optimizing a single objective, prioritizing one aspect of the problem—such as minimizing cost. While effective in some cases, they often fall short when addressing the balance between multiple objectives. Additionally, traditional optimization techniques can be computationally expensive and time-consuming, especially when dealing with large urban areas and numerous potential charging station locations.

Multi-objective optimization techniques, such as multi-objective evolutionary algorithms (MOEAs), offer a promising alternative for addressing the complexity of the electric vehicle charging station infrastructure problem. These algorithms are designed to explore a diverse set of solutions that represent various trade-offs between competing objectives. By considering multiple objectives simultaneously, MOEAs provide decision-makers with a range of optimized solutions, enabling more balanced and informed planning decisions.

\vspace{0.5em}

\textbf{This study aims to answer the following research question: How can multi-objective evolutionary algorithms be applied to optimize electric vehicle charging station infrastructure planning?}

\vspace{0.5em}

To explore this question, this study applies a multi-objective evolutionary algorithm, specifically NSGA-II, to evaluate and optimize different configurations of charging station placement using real-world data. By considering multiple objectives simultaneously, the goal is to develop a balanced and practical approach to EV infrastructure planning. The results aim to support decision-makers in designing charging networks that are efficient and accessible.

\section{Motivation}

The transition to electric vehicles (EVs) is a crucial step toward sustainable transportation, helping to reduce air pollution, greenhouse gas emissions, and reliance on fossil fuels such as gasoline and diesel. As EV adoption accelerates globally, the need for a reliable, accessible, and efficient charging infrastructure becomes increasingly critical.

Designing optimal locations and configurations for electric vehicle charging stations (EVCS) is a complex problem involving multiple, often conflicting objectives. Key goals include maximizing geographic coverage to ensure accessibility for users, enhancing charger speeds to reduce waiting and charging times, minimizing infrastructure costs by limiting the number of stations and chargers, and reducing the average travel distance for EV users to access charging points. Balancing these objectives is essential to foster widespread EV adoption and create a user-friendly charging network.


Traditional optimization methods are often inadequate for solving such complex, multidimensional problems. This is where multi-objective evolutionary algorithms (MOEAs), such as NSGA-II, become valuable. These algorithms are well-suited to handling the complexities of EVCS infrastructure optimization, as they can search for solutions that simultaneously satisfy multiple objectives.

The motivation behind this thesis arises from the need to explore advanced optimization techniques like MOEAs to address the challenges faced by urban planners and decision-makers in developing electric vehicle charging infrastructure. As the number of electric vehicles grows, the demand for a robust charging network increases. The placement and configuration of charging stations must ensure adequate coverage across urban areas while minimizing infrastructure costs.

Furthermore, this research aims to demonstrate how evolutionary algorithms, particularly NSGA-II, can provide efficient and scalable solutions to these challenges, contributing to the creation of a sustainable and efficient electric vehicle charging infrastructure.

\section{Problem Definition}

The growing emphasis on electric vehicles by governments worldwide has created an urgent need for a robust and widespread charging infrastructure. Distributing electric vehicle charging stations optimally across different regions—ensuring high capacity and cost-effectiveness to meet user demand—is a complex challenge. This issue involves balancing multiple objectives, such as expanding coverage, minimizing costs, and reducing waiting times. This research applies multi-objective evolutionary algorithms to address these challenges and identify optimal solutions for electric vehicle charging infrastructure deployment. The problem can be formalized as follows:

\begin{itemize}
    \item \textbf{Maximize Charger Speed:}The efficiency of charging stations plays a key role in determining the speed at which electric vehicles can be recharged. In general, stations with higher power output can reduce charging durations, potentially leading to shorter waiting times for users. These improvements could improve the overall user experience and contribute to a more efficient operation of the charging network. As the adoption of electric vehicles increases, the availability of high speed charging infrastructure becomes an important factor in meeting user needs.
    
    \item \textbf{Maximize Coverage:} Expanding the geographic coverage of electric vehicle charging stations is essential to improve accessibility for users. A well-distributed network can allow electric vehicle owners to find and reach charging points more easily, potentially reducing the need for long detours or extended travel times. Improved accessibility could enhance user convenience and may play a significant role in promoting the adoption of electric vehicles. As more areas are equipped with reliable charging infrastructure, range anxiety could be alleviated, making the ownership of electric vehicles more practical and supporting the transition to a more sustainable transportation system.


    
    \item \textbf{Maximize Charger Speed:} The efficiency of charging stations plays a key role in determining the speed at which electric vehicles can be recharged. In general, stations with higher power output can reduce charging durations, potentially leading to shorter waiting times for users. These improvements may enhance the overall user experience and contribute to a more efficient operation of the charging network. As the adoption of electric vehicles increases, the availability of high-speed charging infrastructure becomes an important factor in meeting user needs.
    
    \item \textbf{Minimize the Number of Stations:} Reducing the number of charging stations, without compromising network coverage, is essential to reduce costs and improve operational efficiency. A careful balance between station location and network coverage ensures that the service area is adequately covered while minimizing infrastructure costs.
    
    \item \textbf{Minimize the Number of Chargers:} In addition to reducing the number of charging stations, it is important to minimize the number of individual chargers at each station. This approach helps lower initial investment and operating costs while still maintaining service levels and user convenience.    

\end{itemize}

However, the economic feasibility of building or expanding electric vehicle charging infrastructure is a significant consideration for decision-makers. Achieving a balance between wide geographic coverage and cost effectiveness is essential to enhancing system performance. This challenge becomes even more critical when coupled with the need to increase charger speeds, which can reduce charging and waiting times and improve overall efficiency.

Although deploying a large number of charging stations in large areas improves accessibility, it requires significant financial investment. This highlights the need for strategic planning to ensure affordability and adequate service coverage for electric vehicle users.

\section{Objective of the Thesis}
This thesis aims to:
    \begin{itemize}
    \item Investigate multi-objective evolutionary algorithms (MOEAs) to solve the EVCS infrastructure problem.
    \item Optimize multiple objectives simultaneously, such as coverage, charger speed (with the goal of upgrading all chargers to Level 3 fast chargers), number of EVCS, number of chargers, and the average distance between EVs and EVCS.
    \item Present and analyze the results from the research experiments conducted in this thesis, and compare them with previous studies.
    \end{itemize}