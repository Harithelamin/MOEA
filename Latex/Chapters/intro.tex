The rapid expansion of electric vehicle adoption presents both an opportunity and a challenge for urban transport systems. Electric vehicles have the potential to reduce air pollution and the use of conventional fuels such as diesel and gasoline, but their widespread use requires the establishment of a reliable and accessible charging infrastructure.

Selecting the location of electric vehicle charging stations is a complex problem that involves balancing several conflicting objectives. As electric vehicle adoption continues to increase worldwide, establishing reliable, efficient, and accessible infrastructure is becoming increasingly important. A key challenge in this process is to identify the optimal locations for charging stations to meet the needs of electric vehicle users while minimizing associated costs and covering as much area as possible to ensure connectivity.

Minimizing overall infrastructure costs an important goal when selecting the location of charging stations. This includes both direct costs, such as installing and maintaining charging stations, and the costs of users accessing those stations. Reducing costs is a critical aspect of urban planning, so it is essential to ensure that charging stations are strategically distributed to keep costs low while still meeting demand. Efficient distribution plays a key role in balancing affordability and user accessibility.

In addition to cost considerations, another important factor in determining the location of EV charging stations is minimizing the travel distance for EV users. The availability of charging stations directly affects the possibility of using EVs. Stations should be strategically placed, whether in densely populated urban areas or rural areas where access to charging infrastructure may be limited. Reducing the distance that drivers have to travel to find a charging point is crucial as it enhances the desire of drivers to use EVs. Conversely, long waits at charging stations or difficulty finding a nearby station can be a reason for drivers to avoid using EVs altogether.

Furthermore, the power level of the chargers should be considered to minimize the charging time. By optimizing the charging power of each station, the overall efficiency of the network can be improved, resulting in faster charging times and a better user experience. This consideration is crucial in ensuring that the charging infrastructure meets the growing demand for electric vehicles while minimizing user waiting times.

Moreover, traditional approaches to solving the EVCS location problem typically rely on optimization techniques such as mathematical programming. This approach often focuses on single-objective optimization, where only one aspect of the problem is prioritized, such as minimizing cost. Although these methods can provide effective solutions in certain scenarios, they often fail when dealing with trade-offs between multiple objectives, such as cost, coverage, and load balancing. In addition, traditional optimization techniques can be computationally expensive and time-consuming, especially when the problem involves large urban areas with many potential charging station locations. Multi-objective optimization techniques, such as multi-objective evolutionary algorithms (MOEAs), have emerged as a promising alternative to address the complexity of the EVCS problem. These algorithms are designed to find a variety of solutions that represent different trade-offs between competing objectives. By incorporating multiple objectives into the optimization process, multi-objective evolutionary algorithms can provide decision makers with a broader set of possible solutions, allowing them to choose the solution that best meets the needs of society while balancing the different costs and benefits of charging station placement. These methods are particularly useful when dealing with real-world problems that require the simultaneous optimization of multiple, often conflicting, objectives. In short, locating electric vehicle charging stations is a complex task that involves addressing different objectives, such as maximizing coverage,  and maximizing the power of the charger to reduce waiting time. Traditional methods may not be sufficient to handle competing objectives, making multi-objective evolutionary algorithms an attractive solution for optimizing EV charging infrastructure placement.

\section{Motivation}

The transition to electric vehicles is a key component of sustainable transportation systems, addressing growing concerns about air pollution, climate change, and reliance on conventional fuels such as gasoline and diesel. As EV adoption continues to rise, one of the most pressing challenges is developing an efficient, widely available, and accessible electric vehicle charging station (EVCS) infrastructure. The success of this transition depends largely on optimizing the placement, configuration, and capacity of charging stations to meet the needs of both private and public transportation. However, this task is complicated by the multiple conflicting objectives that must be balanced, such as minimizing costs, maximizing coverage, and minimizing charging time.

Traditional optimization methods are often inadequate to solve such multidimensional problems. This is where multi-objective evolutionary algorithms (MOEAs), such as NSGA-II, come in. These algorithms are well-suited to handling the complexities of EVCS infrastructure optimization due to their ability to search for solutions that simultaneously meet multiple objectives.

The motivation behind this thesis stems from the need to explore advanced optimization techniques such as MOEAs to address the pressing challenges faced by urban planners and decision makers in the field of electric vehicle charging infrastructure. As the number of electric vehicles grows, the need for a strong charging network increases. The placement and configuration of electric vehicle charging stations must ensure adequate coverage across urban areas, while minimizing infrastructure costs. 

Furthermore, this research aims to demonstrate how evolutionary algorithms, especially NSGA-II, can provide efficient and scalable solutions to these challenges, contributing to the creation of a sustainable and efficient electric vehicle charging infrastructure.


\section{Problem Definition}
The rapid trend of governments to push for electric vehicles has created an urgent need for a strong and widespread electric vehicle charging infrastructure. Ensuring that electric vehicle charging stations are optimally distributed across geographies with high capacity and cost-effectiveness to meet the needs of electric vehicle users is a complex challenge. The problem is complex, involving a number of objectives, such as increasing coverage area, minimizing costs, and reducing waiting time. This research focuses on leveraging multi-objective evolutionary algorithms to address these objectives and find optimal solutions for the deployment of electric vehicle charging infrastructure. The problem can be formalized as follows:
\begin{itemize}
    \item Coverage: The goal is to increase the geographical area covered by electric vehicle charging stations, as the good distribution of electric vehicle charging stations allows electric vehicle owners to easily access these stations, which reduces time and effort.
    
    \item Chargers Power Level: Charger power level: The efficiency of charging stations plays an important role in charging electric vehicles, as it affects the speed of the charging process. Charging stations with high capacity reduce charging time, which leads to less waiting time for users. In addition, reducing charging time enhances the user experience and increases the efficiency of the network in general. It is necessary to provide charging stations capable of handling high power requirements to meet the growing needs of electric vehicles.
    
    \item Reducing the number of stations: Reducing the number of charging stations without compromising network coverage is critical to reducing costs and improving operational efficiency. Balancing station location with network coverage ensures adequate coverage of the service area while minimizing infrastructure costs.
    
    \item Reducing the number of chargers: In addition to reducing the number of charging stations, it is also necessary to reduce the number of individual chargers at each station. This helps reduce initial investment and operating costs while maintaining service levels and user convenience.

        
    \item Cost: Reducing the cost of building electric charging stations is of great importance to decision makers. The balance between comprehensive coverage and affordability is a key element of the optimization problem, but deploying a large number of charging stations across a large area can be very expensive.
\end{itemize}

\section{Objective of the Thesis}
This thesis aims to:
\begin{itemize}
    \item Investigating multi-objective evolution algorithms (MOEAs) to solve the EVCS infrastructure problem.
    \item Optimizing multiple objectives at the same time, including infrastructure cost, coverage, and power levels per charging stand.
    \item Comparing the results obtained from the application of the genetic algorithm with those found in the literature review.
    \item Presenting the results from the research thesis experiments and analyzing their performance.
\end{itemize}

